\documentclass[t,handout]{beamer}
\usepackage[default]{sourcesanspro}
% \usepackage[default]{sourcecodepro}
% \usepackage[default]{sourceserifpro}
\usepackage[utf8]{inputenc}
\usepackage[english]{babel}
\usepackage{listings}
\lstset{ %
  basicstyle=\ttfamily\color{black}\tiny,
  breaklines=true,
  columns=fullflexible,
  frame=single,
  keepspaces=true,
  tabsize=2
}

\definecolor{darkblue}{rgb}{0,0,.5}
\hypersetup{pdftex=true, colorlinks=true, breaklinks=true, linkcolor=darkblue, menucolor=darkblue, pagecolor=darkblue, urlcolor=darkblue}

\title{Taskwarrior -- What's next?}
\subtitle{Introduction to Taskwarrior}
\author[Deimeke, Dirk]{Dirk Deimeke}
\institute[Taskwarrior academy]{Taskwarrior academy}
\date{FrOSCon 2016}
\titlegraphic{\includegraphics[width=3cm,height=3cm]{tw-xxl}}
\subject{Taskwarrior}
\keywords{taskwarrior, task, management, commandline, getting things done}

\setbeamercovered{transparent}

\pgfdeclareimage[width=5mm]{tw-logo}{tw-xl}

%%%%%%%%%%%%%%%%%%%%%%%%%%%%%%%%%%%%%%%%%%%%%%%%%%%%%%%%%%%%%%%%%%%%%%%
%
% LaTeX-Template for Taskwarrior by Dominik Wagenführ
% http://www.deesaster.org/
%
% Creative Commons Attribution-ShareAlike 4.0 (CC-BY-SA 4.0)
% https://creativecommons.org/licenses/by-sa/4.0/
%
%%%%%%%%%%%%%%%%%%%%%%%%%%%%%%%%%%%%%%%%%%%%%%%%%%%%%%%%%%%%%%%%%%%%%%%

%%%%%%%%%%%%%%%%%%%%%%%%%%%
% Layout - Start
%%%%%%%%%%%%%%%%%%%%%%%%%%%

\setbeamertemplate{navigation symbols}{}

\useinnertheme{default}
\useoutertheme{infolines}
\usefonttheme{structurebold}

\newlength{\boxwidth}
\setlength{\boxwidth}{121px}
\setbeamertemplate{headline}{%
\begin{beamercolorbox}[dp=3px,ht=6px,wd=\boxwidth,center]{palette tertiary}%
\insertauthor\ (\insertinstitute)%
\end{beamercolorbox}%
\vskip-9px\hskip\boxwidth
\begin{beamercolorbox}[dp=3px,ht=6px,wd=\boxwidth,center]{palette secondary}%
\inserttitle%
\end{beamercolorbox}%
\begin{beamercolorbox}[dp=3px,ht=6px,wd=\boxwidth,right]{palette primary}%
\insertdate\hskip15px\insertframenumber\,/\,\inserttotalframenumber\hspace*{8px}
\end{beamercolorbox}%
}
\setbeamertemplate{footline}{}

%%%%%%%%%%%%%%%%%%%%%%%%%%%
% Colors - Start
%%%%%%%%%%%%%%%%%%%%%%%%%%%

\definecolor{basecolor}{gray}{0.2}
\definecolor{mittelgrau}{gray}{0.4}
\definecolor{hellgrau}{gray}{0.93}
\definecolor{gelb}{rgb}{1.0,1.0,0.75}

% infolines color
\setbeamercolor{palette primary}{fg=white,bg=mittelgrau}
\setbeamercolor{palette secondary}{fg=white,bg=basecolor}
\setbeamercolor{palette tertiary}{fg=white,bg=black}

% frame and title color
\setbeamercolor{frametitle}{fg=white,bg=basecolor}
\setbeamercolor{titlelike}{fg=white,bg=basecolor}
\setbeamerfont{frametitle}{series=\bfseries}

% TOC color
\setbeamercolor{section in toc}{fg=basecolor}

% text color
\setbeamercolor{normal text}{fg=black,bg=white}

% item color
\setbeamercolor{item}{fg=basecolor}
\setbeamercolor{itemize item}{fg=black}

% block color
\setbeamercolor{block title}{fg=black,bg=gelb}
\setbeamercolor{block body}{fg=black,bg=gelb}

% block alert color
\setbeamercolor{block title alerted}{fg=black,bg=gelb}
\setbeamercolor{block body alerted}{fg=black,bg=gelb}

% block example color
\setbeamercolor{block title example}{fg=black,bg=gelb}
\setbeamercolor{block body example}{fg=black,bg=gelb}

\setbeamertemplate{frametitle}{%
\vskip-1px%
\begin{beamercolorbox}[wd=363px,ht=20px,dp=12px]{frametitle}
\usebeamerfont{frametitle}%
\usebeamercolor{frametitle}%
\vspace*{-5px}\hspace*{10px}\includegraphics[width=24px,height=24px]{tw-xl.png}\\
\vspace*{-20px}\hspace*{40px}\insertframetitle%
\end{beamercolorbox}
}

%%%%%%%%%%%%%%%%%%%%%%%%%%%
% Colors - Stop
%%%%%%%%%%%%%%%%%%%%%%%%%%%

%%%%%%%%%%%%%%%%%%%%%%%%%%%%%%%%%%%%%%%%%%%%%%%%%%%%%%%%%%%%%%%%%%%%%%%

\begin{document}

\begin{frame} % Titel
	\titlepage
\end{frame}

% \logo{\pgfuseimage{tw-logo}}

\begin{frame}\frametitle{Content}
	\tableofcontents
\end{frame}

%%%%%%%%%%%%%%%%%%%%%%%%%%%
\section{Introduction}
%%%%%%%%%%%%%%%%%%%%%%%%%%%

\parskip1ex

\begin{frame}[fragile]\frametitle{Dirk Deimeke (that's me)}
    \begin{itemize}
        \item Born 1968 in Wanne-Eickel
        \item Linux since 1996
        \item Emigrated 2008 to Switzerland
        \item Taskwarrior Team since 2010
    \end{itemize}

    Entry point for more \url{https://d5e.org/}
\end{frame}

\begin{frame}\frametitle{Project founder: Paul Beckingham}
    \pause
    \begin{itemize}
        \item I started out using Gina Trapani's todo.sh, which was great, but I soon wanted features that would have been difficult to implement in a shell script, so I wrote my own. \pause
        \item It stemmed from the fact that a todo program needs to be simple to use, and unobtrusive, otherwise it's a hassle. But it can't be too simple. \pause
        \item If you go to the trouble of capturing this information, it seems wasteful not to leverage it. So it has a lot of features, but tries to remain simple to use. \pause
        \item There are many different methodologies people use for managing their work, and Taskwarrior tries to walk a line through the middle of all that, with features for all the different approaches. \pause
        \item Taskwarrior is intended to scale with the user, from very simple straightforward usage up to quite sophisticated task management.
    \end{itemize}
\end{frame}

\begin{frame}[fragile]\frametitle{Reasons for Taskwarrior}
Taskwarrior
\begin{itemize}
	\item is easy to learn.
	\item grows along with the work.
	\item is unbelievably powerful.
	\item is very fast.
	\item is easily extensible.
	\item is platform independent:
	\begin{itemize}
		\item Most flavours of Unix and Linux, including Mac OS X
        \item Windows with Cygwin (unsupported, but working)
		\item Android with Termux
        \item Third-Party Apps (Android-Client, GUI based on NodeJS)
	\end{itemize}
	\item is actively developed.
	\item can be influenced by users (feature requests).
	\item has excellent and very friendly support.
\end{itemize}
\end{frame}

\begin{frame}[fragile]\frametitle{History -- Some milestones}
    \begin{description}
        \item[2006-11-29] Version 0.0.1, Project began as enhancement to todo.txt.
        \item[2008-06-03] Version 1.0.0 released
        \item[2012-03-17] Version 2.0.0 released
        \item[2014-01-15] Version 2.3.0, Task Server support
        \item[2015-10-21] Version 2.5.0, Improved command line parser
        \item[2016-02-24] Version 2.5.1 is a bug fix, code cleanup, performance release only -- no features.
        \item[?]          Version 2.6.0 will overhaul recurrence and add more flavors of recurring tasks.
    \end{description}

    \url{http://taskwarrior.org/docs/history.html}
\end{frame}

%%%%%%%%%%%%%%%%%%%%%%
\section{Installation}
%%%%%%%%%%%%%%%%%%%%%%

\begin{frame}[fragile]\frametitle{Installation from source}

    \begin{alertblock}{Attention!}
    Since some packagers (Ubuntu as example) implement their thinking of the place where files have to be without changing the templates, an installation from source is the recommended way.
    \end{alertblock}
    \pause

    All you need to compile is
    \begin{itemize}
        \item \verb+GnuTLS+ (ideally version 3.2 or newer)
        \item \verb+libuuid+
        \item \verb+CMake+ (2.8 or newer)
        \item \verb+make+
        \item A C++ Compiler (\verb+GCC 4.7+ or \verb+Clang 3.3+ or newer)
    \end{itemize}

\end{frame}

% \pause

%\begin{frame}\frametitle{5 steps to install from source}
%\begin{enumerate}
%\item \href{http://taskwarrior.org/download}{Download} of recent version
%\item \texttt{tar xzf task-2.0.0.beta4.tar.gz}
%\item \texttt{cmake .} (not \texttt{./configure}!) or \newline
%      \texttt{cmake -DCMAKE_INSTALL_PREFIX=/home/user/task}
%\item \texttt{make}
%\item \texttt{sudo make install}
%\end{enumerate}
%\end{frame}
%\frametitle{Installation with package management}

%%%%%%%%%%%%%%%%%%%%%%%%%%%
\section{Simple ToDo-Lists}
%%%%%%%%%%%%%%%%%%%%%%%%%%%

%\frametitle{A simple example}
%\frametitle{Commands so far}
%\frametitle{That's it!}

%%%%%%%%%%%%%%%%%
\section{General}
%%%%%%%%%%%%%%%%%

%\frametitle{Just kidding ...}
%\frametitle{task {\tt<}filter{\tt>} command {\tt<}mods{\tt>}}
%\frametitle{Scripts}
%\frametitle{Most important commands}

%%%%%%%%%%%%%%%%%%%%%%%%%%%%
\section{Working with dates}
%%%%%%%%%%%%%%%%%%%%%%%%%%%%

%\frametitle{Dateformats -- from 'man taskrc'}
%\frametitle{Set dateformat}
%\frametitle{Set weekstart}
%\frametitle{Special dates}
%\frametitle{Due and wait}
%\frametitle{Recurrence}
%\frametitle{Recurrence modifiers}
%\frametitle{Until and entry}
%\frametitle{Starting and stopping}
%\frametitle{Holiday}
%\frametitle{Add holiday}
%\frametitle{Calendar config with holiday}
%\frametitle{Calendar with holiday}
%\frametitle{Calendar with due tasks}
%\frametitle{Timesheet}

%%%%%%%%%%%%%%%%%%%%%%%%
\section{Getting sorted}
%%%%%%%%%%%%%%%%%%%%%%%%

%\frametitle{Project and subproject}
%\frametitle{Projects}
%\frametitle{Tags}
%\frametitle{Annotations}

%%%%%%%%%%%%%%%%%%%%%%
\section{Dependencies}
%%%%%%%%%%%%%%%%%%%%%%

%\frametitle{Dependency}

%%%%%%%%%%%%%%%%%
\section{Reports}
%%%%%%%%%%%%%%%%%

%\frametitle{Predefined reports (from task reports)}
%\frametitle{burndown.daily}
%\frametitle{ghistory, history}
%\frametitle{ls, minimal, summary}
%\frametitle{Report definitions}
%\frametitle{Set default command}

%%%%%%%%%%%%%%%%%%%
\section{Filtering}
%%%%%%%%%%%%%%%%%%%

%\frametitle{Filtering in general}
%\frametitle{Searches}
%\frametitle{Attribute modifiers}
%\frametitle{Combining}
%\frametitle{Or ...}
%\frametitle{Search configuration}
%\frametitle{Filter in reports}

%%%%%%%%%%%%%%%%%%%%%%
\section{Miscellanous}
%%%%%%%%%%%%%%%%%%%%%%

%\frametitle{This is by far not all}

%%%%%%%%%%%%%%%%%%%%
\section{That's all}
%%%%%%%%%%%%%%%%%%%%

%\frametitle{Questions?}
%\frametitle{Support}
%\frametitle{Thanks for your patience!}

\end{document}

% \begin{frame}[fragile]\frametitle{Title}
% \begin{lstlisting}
% \end{lstlisting}
% \end{frame}
%
% \begin{frame}\frametitle{title}
% \begin{center}
% \includegraphics[width=10cm,height=7.5cm]{name.png}
% \end{center}
% \end{frame}
%
% \begin{frame}\frametitle{title}
% \begin{itemize}
% \item \textbf{task {\tt<}filter{\tt>} modify}
% \end{itemize}
% \end{frame}
