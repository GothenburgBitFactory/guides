% \documentclass[t,handout,aspectratio=169]{beamer}
% \documentclass[t,aspectratio=169]{beamer}
% \documentclass[t,handout]{beamer}
\documentclass[t,aspectratio=169]{beamer}
\usetheme[
    sectionpage=progressbar,
    numbering=none,
    block=fill,
]{metropolis}

\usepackage[T1]{fontenc} % for copy'n'paste
\usepackage[utf8]{inputenc}
% \usepackage[english,nswissgerman]{babel}
\usepackage[nswissgerman,english]{babel}
\usepackage{listings}
\lstset{ %
    basicstyle=\ttfamily\scriptsize,
    breaklines=true,
    columns=fullflexible,
    keepspaces=true,
    language=bash,
    tabsize=2
}

% black, blue, brown, cyan, darkgray, gray, green, lightgray, lime, magenta, olive, orange, pink, purple, red, teal, violet, white, yellow.
%\hypersetup{ %
%    colorlinks=true,
%    linkcolor=,
%    urlcolor=orange
%}

%%%%%%%%%%%%%%%%%%%%%%%%%%%%%%%%%%%%%%%%%%%%%%%%%%%%%%%%%%%%%%%%%%%%%%%

\title{Lessons Learned}
\subtitle{From Running \\ an Open Source Project \\ for more than a decade.}
\titlegraphic{\hfill\includegraphics[height=3cm]{tw-xl.png}}

\date{May, 23rd 2017}
\author{Dirk Deimeke}
\institute{Taskwarrior Academy @ LinuxERFA}

\subject{Taskwarrior}
\keywords{Taskwarrior}

%%%%%%%%%%%%%%%%%%%%%%%%%%%%%%%%%%%%%%%%%%%%%%%%%%%%%%%%%%%%%%%%%%%%%%%
\begin{document}

\begin{frame}
    \titlepage
\end{frame}

%%%%%%%%%%%%%%%%%%%%%%%%%%%%%%%%%%%%%%%%%%%%%%%%%%%%%%%%%%%%%%%%%%%%%%%%
\section{Prolog}
%%%%%%%%%%%%%%%%%%%%%%%%%%%%%%%%%%%%%%%%%%%%%%%%%%%%%%%%%%%%%%%%%%%%%%%%

\parskip1ex

\begin{frame}[standout]
    Taskwarrior -- \href{https://taskwarrior.org/}{taskwarrior.org}
\end{frame}

\begin{frame}[fragile]\frametitle{Taskwarrior Philosophy}
    \vfill
    \begin{itemize}
        \item Openness
        \item Low Friction
        \item No Penalty
        \item Methodology Agnostic
        \item Toolkit
        \item Extension Friendly
        \item Community
        \item Focus
    \end{itemize}

    \href{https://taskwarrior.org/docs/philosophy.html}{taskwarrior.org/docs/philosophy.html}
\end{frame}

%%%%%%%%%%%%%%%%%%%%%%%%%%%%%%%%%%%%%%%%%%%%%%%%%%%%%%%%%%%%%%%%%%%%%%%%
\section{Lessons learned}
%%%%%%%%%%%%%%%%%%%%%%%%%%%%%%%%%%%%%%%%%%%%%%%%%%%%%%%%%%%%%%%%%%%%%%%%

\begin{frame}[fragile]\frametitle{What Have We Learned From This Open Source Project?}
    \vfill
    Here is the collected \textit{wisdom} that we have gained from \\
    running the Taskwarrior project for more than a decade. \pause

    It has been \textbf{rewarding}, \textbf{enjoyable}, and sometimes \textbf{frustrating}. \pause

    We learned a lot about users and Open Source expectations.
\end{frame}

\begin{frame}[standout]
    I will speak about our experiences \\
    and would like to hear yours.
\end{frame}

\begin{frame}[fragile] % \frametitle{Motivation}
    \vfill
    \onslide<1->{\textbf{Start an open source project if you want to learn all you can}} \onslide<4->{about software design, development, planning, testing, documenting, and delivery;} \onslide<2->{\textbf{enjoy technical challenges}}\onslide<4->{, administrative challenges, compromise, and will} \onslide<3->{\textbf{be satisfied hoping that someone out there is benefitting from your work}.}
\end{frame}

\begin{frame}[fragile] % \frametitle{Demotivation}
    \vfill
    \onslide<1->{Do \textbf{not} start an open source project} \onslide<2->{if you need praise, warmth and love from your fellow human beings.}
\end{frame}

\begin{frame}[fragile] % \frametitle{Boundaries}
    \vfill
    If you could draw a boundary between that which is already supported, and that which is not, \pause you would find that all the activity, discussion and drama occurs at that boundary. \pause

    Feature requests only nibble at the periphery. \pause

    Bold changes originate elsewhere.
\end{frame}

\begin{frame}[fragile]% \frametitle{x}
    \vfill
    People will get excited about something a project doesn't yet support. \pause

    Deliver it, and they will get excited about the next thing.
\end{frame}

\begin{frame}[fragile]% \frametitle{x}
    \vfill
    If a feature works well, you'll never hear about it again.
\end{frame}

\begin{frame}[fragile]% \frametitle{x}
    \vfill
    There is a fine line between \textit{richly-featured} and \textit{bloated}. \pause

    There may not be a line at all.
\end{frame}

\begin{frame}[fragile]% \frametitle{x}
    \vfill
    If you demo two features, and talk about twenty more, \\
    users still only know about the two. \pause

    Visual demonstrations have far greater impact.
\end{frame}

\begin{frame}[fragile]% \frametitle{x}
    \vfill
    Every change will ruin someone's day.

    They will be sure to tell you about it. \pause

    The same change will improve someone's day.

    You will not hear of this.
\end{frame}

\begin{frame}[fragile]% \frametitle{x}
    \vfill
    People will disguise feature requests as bugs, \pause \\
    which means either they consider difference of opinion a defect, \pause \\
    or believe that calling it a flaw will force implementation, \pause \\
    but hopefully they just forgot to set the issue type to \textit{enhancement}.
\end{frame}

\begin{frame}[fragile]% \frametitle{x}
    \vfill
    Some people find it very difficult to articulate what they want. \pause

    It's worth being patient and finding out what they need.
\end{frame}

\begin{frame}[fragile]% \frametitle{x}
    \vfill
    What you keep out of a project \pause \\
    is just as important as what you allow in to a project.
\end{frame}

\begin{frame}[fragile]% \frametitle{x}
    \vfill
    Many new users will submit feature requests, \\
    just to show that they are knowledgeable and clever. \pause

    They don't really want that feature, it's a form of positive feedback.
\end{frame}

\begin{frame}[fragile]% \frametitle{x}
    \vfill
    Beware of suggestions from users \\
    who have used your software for only a day or so. \pause

    Be equally aware of suggestions from users \\
    who have used your software for a long, long time.
\end{frame}

\begin{frame}[fragile]% \frametitle{x}
    \vfill
    People will threaten to not use open source software because it lacks a feature, \pause \\
    thereby mistaking themselves for paying customers.
\end{frame}

\begin{frame}[fragile]% \frametitle{x}
    \vfill
    Many believe that if a change is small, it deserves to be in the project, \pause \\
    regardless of whether it makes sense for it to be there.
\end{frame}

\begin{frame}[fragile]% \frametitle{x}
    \vfill
    Users will go to the effort of seeking you out online, \pause \\
    to directly ask you a question \pause \\
    that is answered two clicks from the front page of a website.
\end{frame}

\begin{frame}[fragile]% \frametitle{x}
    \vfill
    A looping, animated GIF will be watched over and over, scrutinized and understood.

    \pause A paragraph of text will be ignored.
\end{frame}

\begin{frame}[fragile]% \frametitle{x}
    \vfill
    Man pages are too densely crammed with information, \\
    and too lengthy, for most modern humans to ingest.
\end{frame}

\begin{frame}[fragile]% \frametitle{x}
    \vfill
    The best question to identify time wasters: \pause

    \textit{What have you tried so far?}
\end{frame}

\begin{frame}[fragile]% \frametitle{x}
    \vfill
    People will pick a fight with you about all your incidental choices. \pause

    Your issue tracker, \\
    your branching strategy, \\
    your version numbers, \\
    the text editor you use, \\
    and so on.
\end{frame}

\begin{frame}[fragile]% \frametitle{x}
    \vfill
    You can choose the most permissive software license, \\
    and people will still argue with you about your choice.
\end{frame}

\begin{frame}[fragile]% \frametitle{x}
    \vfill
    SEO consultants are not very good at searching the web, \pause \\
    and learning that you operate an open source, non-profit project. \pause

    It says a lot.
\end{frame}

\begin{frame}[fragile]% \frametitle{x}
    \vfill
    No one has ever complained about an algorithm choice, \\
    code structure, or code comments, \pause \\
    but dozens have told us that our use of whitespace is wrong. \pause

    Complaints have not been about the code, but the gaps between code. \pause

    Prioritize the complaints.
\end{frame}

\begin{frame}[fragile]% \frametitle{x}
    \vfill
    We hard-coded XTerm color control sequences, bypassing termcap.

    That was more than ten years ago.

    No one has noticed. \pause

    Sometimes, what looks like an expedient shortcut is perfectly good.
\end{frame}

\begin{frame}[fragile]% \frametitle{x}
    \vfill
    We had a very long and detailed tutorial page on the site for years.

    To go through and read it all would have taken at least an hour. \pause

    At the very bottom, was a video of a band playing the \\
    Mexican Hat Dance using only hand-fart noises. \pause

    No one ever mentioned this.

    Keep your tutorials short.
\end{frame}

\begin{frame}[fragile]% \frametitle{x}
    \vfill
    Presence at industry events is important. \pause

    Offering talks and workshops helps make people aware of your project.
\end{frame}

\begin{frame}[fragile]% \frametitle{x}
    \vfill
    \textit{Virtual teams} work well, \pause \\
    but it gets even better after meeting in real life.
\end{frame}

\begin{frame}[fragile]% \frametitle{x}
    \vfill
    It is good if the members of your team share the same sense of humor. \pause

    If not, be careful writing messages with an ironic tone.
\end{frame}

\begin{frame}[fragile]% \frametitle{x}
    \vfill
    A development-class machine is no indication \\
    of the kind of hardware and software your users are running. \pause

    Dependencies and tools are often far behind the latest versions.
\end{frame}

\begin{frame}[fragile]% \frametitle{x}
    \vfill
    Respond to every means of communication.

    It is worth it.
\end{frame}

\begin{frame}[fragile]% \frametitle{x}
    \vfill
    Have a recognizable logo. \pause

    Do not make the logo yourself, if you are not a designer. \pause

    If you have no budget, ask a designer to judge your work.
\end{frame}

\begin{frame}[fragile]% \frametitle{x}
    \vfill
    Offering gratis stickers is great, \pause \\
    having SWAG -- Souvenirs, Wearables And Gifts -- users can choose from is even better.
\end{frame}

\begin{frame}[fragile]% \frametitle{x}
    \vfill
    People love to make mashups of two things, or add an extension to a thing. \pause

    Very few contributors want to work on the thing.
\end{frame}

\begin{frame}[fragile]% \frametitle{x}
    \vfill
    Calm down, take a deep breath and look back at what you achieved. \pause

    Details, mistakes, compromises, incomplete plans and unfulfilled wishes \\
    are only visible from inside the project. Be proud, and make new plans.
\end{frame}

\begin{frame}[fragile]% \frametitle{x}
    \vfill
    Create a website containing the philosophy behind your project \\
    to help people understand what your project is about.
\end{frame}

%%%%%%%%%%%%%%%%%%%%%%%%%%%%%%%%%%%%%%%%%%%%%%%%%%%%%%%%%%%%%%%%%%%%%%%%
\section{Epilog}
%%%%%%%%%%%%%%%%%%%%%%%%%%%%%%%%%%%%%%%%%%%%%%%%%%%%%%%%%%%%%%%%%%%%%%%%

\begin{frame}[fragile]\frametitle{That's all!}
    \vfill
    \begin{center}
        Dirk Deimeke, Taskwarrior-Team, 2017, \href{https://creativecommons.org/licenses/by/4.0/}{CC-BY}

        \href{mailto:dirk@deimeke.net}{dirk@deimeke.net} -- \href{https://d5e.org/}{d5e.org}
    \end{center}
\end{frame}

\end{document}
