% \documentclass[t,handout,aspectratio=169]{beamer}
% \documentclass[t,aspectratio=169]{beamer}
\documentclass[t,handout]{beamer}
% \documentclass[t]{beamer}
\usetheme[
    sectionpage=progressbar,
    numbering=none,
    block=fill,
]{metropolis}

\usepackage[T1]{fontenc} % for copy'n'paste
\usepackage[utf8]{inputenc}
% \usepackage[english,nswissgerman]{babel}
\usepackage[nswissgerman,english]{babel}
\usepackage{listings}
\lstset{ %
    basicstyle=\ttfamily\scriptsize,
    breaklines=true,
    columns=fullflexible,
    keepspaces=true,
    language=bash,
    tabsize=2
}

%%%%%%%%%%%%%%%%%%%%%%%%%%%%%%%%%%%%%%%%%%%%%%%%%%%%%%%%%%%%%%%%%%%%%%%

\title{Taskwarrior}
\subtitle{From Source to Getting Things Done}
\titlegraphic{\hfill\includegraphics[height=3cm]{tw-xl.png}}

\date{November, 6th 2016}
\author{Dirk Deimeke}
\institute{Taskwarrior Academy / OpenRheinRuhr 2016}

\subject{Taskwarrior}
\keywords{Taskwarrior}

%%%%%%%%%%%%%%%%%%%%%%%%%%%%%%%%%%%%%%%%%%%%%%%%%%%%%%%%%%%%%%%%%%%%%%%
\begin{document}

\begin{frame}
    \titlepage
\end{frame}

%\begin{frame}[fragile]\frametitle{Table of contents}
%    \vfill
%    \setcounter{tocdepth}{1}
%    \tableofcontents
%\end{frame}

%    \begin{block}{Normaler Block}		% normaler Block
%    \end{block}

%    \begin{exampleblock}{Beispielblock}	% Beispielblock
%    \end{exampleblock}

%    \begin{alertblock}{Warnblock}		% Warnblock
%    \end{alertblock}

%%%%%%%%%%%%%%%%%%%%%%%%%%%%%%%%%%%%%%%%%%%%%%%%%%%%%%%%%%%%%%%%%%%%%%%%
\section{Prolog}
%%%%%%%%%%%%%%%%%%%%%%%%%%%%%%%%%%%%%%%%%%%%%%%%%%%%%%%%%%%%%%%%%%%%%%%%

\parskip1ex

\begin{frame}[standout]
    Dirk Deimeke -- \href{https://d5e.org/}{d5e.org}
\end{frame}

\begin{frame}[fragile]\frametitle{Project founder: Paul Beckingham}
    \vfill
    \pause
    \begin{itemize}
        \item I started out using Gina Trapani's todo.sh, which was great, but I soon wanted features that would have been difficult to implement in a shell script, so I wrote my own. \pause
        \item There are many different methodologies people use for managing their work, and Taskwarrior tries to walk a line through the middle of all that, with features for all the different approaches. \pause
        \item Taskwarrior is intended to scale with the user, from very simple straightforward usage up to quite sophisticated task management. \pause
        \item \textbf{It has a lot of features, but tries to remain simple to use.}
    \end{itemize}
\end{frame}

\begin{frame}[fragile]\frametitle{Reasons pro Taskwarrior}
    \vfill
    \textbf{Taskwarrior} \pause
    \begin{itemize}
    	\item is easy to learn. \pause
    	\item grows along with the work. \pause
    	\item is unbelievably powerful. \pause
    	\item is very fast. \pause
    	\item is easily extensible. \pause
    	\item is platform independent:
    	\begin{itemize}
    		\item Most flavours of Unix and Linux, including Mac OS X
            \item Windows 10 Linux Subsystem \\
                Other Windows versions with Cygwin (unsupported)
    		\item Android with Termux
            \item Third-Party Apps (Android-Client, GUI based on NodeJS, \ldots)
    	\end{itemize} \pause
    	\item is actively developed. \pause
    	\item can be influenced by users (feature requests). \pause
    	\item has excellent and very friendly support.
    \end{itemize}
\end{frame}

\begin{frame}[fragile]\frametitle{This workshop \ldots}
    \vfill

    \begin{alertblock}{This workshop hopefully is a \textit{real workshop}.}
        It will live from you doing things and asking, \\
        it is not about me talking all of the time.
    \end{alertblock}

    {\tiny Nevertheless I will show you every command.}
\end{frame}

%%%%%%%%%%%%%%%%%%%%%%%%%%%%%%%%%%%%%%%%%%%%%%%%%%%%%%%%%%%%%%%%%%%%%%%%
\section{Installation}
%%%%%%%%%%%%%%%%%%%%%%%%%%%%%%%%%%%%%%%%%%%%%%%%%%%%%%%%%%%%%%%%%%%%%%%%

\begin{frame}[fragile]\frametitle{Installation from source}

    \begin{alertblock}{Attention!}
        Since some packagers (Debian and Ubuntu as examples) implement their thinking of the place where files have to be without changing the templates, an installation from source is the recommended way.
    \end{alertblock} \pause

    All you need to compile is
    \begin{itemize}
        \item \verb+GnuTLS+ (ideally version 3.2 or newer)
        \item \verb+libuuid+ \\
            {\small (Darwin, FreeBSD \ldots include \verb+libuuid+ functionality in libc.)}
        \item \verb+CMake+ (2.8 or newer)
        \item \verb+make+
        \item C++ Compiler supporting C++11 \\
            {\small (\verb+GCC 4.7+ or \verb+Clang 3.3+ or newer)}
    \end{itemize}
\end{frame}

\begin{frame}[fragile]\frametitle{Install dependencies}
    Install the necessary packages with your package manager.

    \begin{description}
        \item[CentOS, Fedora, openSUSE] \hfill \\
            gnutls-devel libuuid-devel cmake gcc-c++ \# or clang
        \item[Debian, Ubuntu] \hfill \\
            libgnutls28-dev uuid-dev cmake g++ \# or clang
        \item[Mac OS X] \hfill \\
            Install Xcode from Apple, via the AppStore, launch it, and select from some menu that you want the command line tools.

            With \href{http://brew.sh/}{Homebrew} install the necessary packages:

            \verb=brew install cmake git gnutls=
    \end{description}

\end{frame}

\begin{frame}[fragile]\frametitle{Get the source}
    \vfill
    Either

    \begin{lstlisting}
curl -LO http://taskwarrior.org/download/task-2.5.1.tar.gz
tar xzf task-2.5.1.tar.gz
cd task-2.5.1\end{lstlisting} \pause

    or

    \begin{lstlisting}
git clone --recursive https://git.tasktools.org/scm/tm/task.git task.git
cd task.git

# Updates
git pull --all --recurse-submodules=yes
# git submodule init (if first time)
git submodule update\end{lstlisting}
\end{frame}

\begin{frame}[fragile]\frametitle{Recent development version}
    \vfill
    \begin{lstlisting}
git clone --recursive https://git.tasktools.org/scm/tm/task.git task.git
cd task.git

git checkout 2.6.0

# Magic happening here
git submodule init
git submodule update\end{lstlisting}
\end{frame}

\begin{frame}[fragile]\frametitle{Compile it}
    \vfill
    Three easy steps \ldots hopefully! \pause
    \begin{enumerate}
        \item \verb+cmake -DCMAKE_BUILD_TYPE=release .+ \\
              add the following before the dot (if necessary) \\
              \verb+-DCMAKE_INSTALL_PREFIX=/home/user/task+ \pause
        \item \verb=make= \\
            {\small \verb+export MAKEFLAGS="-j 2"+ to speed things up (number of CPUs)} \pause
        \item \verb=sudo make install=
    \end{enumerate}
\end{frame}

\begin{frame}[fragile]\frametitle{Test it}
    \vfill
    \begin{lstlisting}
$ task diagnostics
A configuration file could not be found in

Would you like a sample /home/dirk/.taskrc created, so Taskwarrior can proceed? (yes/no) yes

task 2.6.0
   Platform: Linux

Compiler
    Version: 6.2.1 20160916 (Red Hat 6.2.1-2)
       Caps: +stdc +stdc_hosted +LP64 +c8 +i32 +l64 +vp64 +time_t64
 Compliance: C++11

Build Features
      Built: Nov  3 2016 14:14:33
     Commit: bcfebff
     CMake: 3.6.2
   libuuid: libuuid + uuid_unparse_lower
 libgnutls: 3.5.5
Build type: release

Configuration
      File: /home/dirk/.taskrc (found), 1465 bytes, mode 100664
      Data: /home/dirk/.task (found), dir, mode 40755
   Locking: Enabled
        GC: Enabled
    Server:
     Trust: strict
Certificate: , not readable, 0 bytes
       Key: , not readable, 0 bytes
   Ciphers: NORMAL
     Creds:

Hooks
    System: Enabled
  Location: /home/dirk/.task/hooks
            (-none-)

Tests
     $TERM: xterm-256color (144x43)
      Dups: Scanned 0 tasks for duplicate UUIDs:
            No duplicates found
Broken ref: Scanned 0 tasks for broken references:
            No broken references found\end{lstlisting}
\end{frame}

%%%%%%%%%%%%%%%%%%%%%%%%%%%%%%%%%%%%%%%%%%%%%%%%%%%%%%%%%%%%%%%%%%%%%%%%
\section{Simple ToDo-Lists}
%%%%%%%%%%%%%%%%%%%%%%%%%%%%%%%%%%%%%%%%%%%%%%%%%%%%%%%%%%%%%%%%%%%%%%%%

\begin{frame}[fragile]\frametitle{A simple example}
    \vfill
    \begin{lstlisting}
task add

task list

task <ID> start

task list

task <ID> stop

task list

task <ID> done\end{lstlisting}
\end{frame}

%%%%%%%%%%%%%%%%%%%%%%%%%%%%%%%%%%%%%%%%%%%%%%%%%%%%%%%%%%%%%%%%%%%%%%%%
\section{Choose a theme}
%%%%%%%%%%%%%%%%%%%%%%%%%%%%%%%%%%%%%%%%%%%%%%%%%%%%%%%%%%%%%%%%%%%%%%%%

\begin{frame}[fragile]\frametitle{Theme}
    \vfill
    Uncomment the theme you want to use from \verb+~/.taskrc+
    \begin{lstlisting}
# Color theme (uncomment one to use)
#include /usr/local/share/doc/task/rc/light-16.theme
#include /usr/local/share/doc/task/rc/light-256.theme
#include /usr/local/share/doc/task/rc/dark-16.theme
#include /usr/local/share/doc/task/rc/dark-256.theme
#include /usr/local/share/doc/task/rc/dark-red-256.theme
#include /usr/local/share/doc/task/rc/dark-green-256.theme
#include /usr/local/share/doc/task/rc/dark-blue-256.theme
#include /usr/local/share/doc/task/rc/dark-violets-256.theme
#include /usr/local/share/doc/task/rc/dark-yellow-green.theme
#include /usr/local/share/doc/task/rc/dark-gray-256.theme
#include /usr/local/share/doc/task/rc/dark-gray-blue-256.theme
#include /usr/local/share/doc/task/rc/solarized-dark-256.theme
include /usr/local/share/doc/task/rc/solarized-light-256.theme
#include /usr/local/share/doc/task/rc/no-color.theme\end{lstlisting}
\end{frame}

\begin{frame}[fragile]\frametitle{Theme}
    \vfill
    \begin{alertblock}{Packaged Taskwarrior}
        Your package distributor might have different ideas where the theme files should be.

        Check with \verb+find / -name no-color.theme -type f 2>/dev/null+
    \end{alertblock}
\end{frame}

%%%%%%%%%%%%%%%%%%%%%%%%%%%%%%%%%%%%%%%%%%%%%%%%%%%%%%%%%%%%%%%%%%%%%%%%
\section{General}
%%%%%%%%%%%%%%%%%%%%%%%%%%%%%%%%%%%%%%%%%%%%%%%%%%%%%%%%%%%%%%%%%%%%%%%%

\begin{frame}[fragile]\frametitle{Nearly all commands work on a bunch of tasks}
    \vfill
    \textbf{There is a lot more to explore.}

    Even the commands from the last section are more mighty than they seem. \pause

    \begin{itemize}
        \item task add {\tt<}mods{\tt>}
        \item task {\tt<}filter{\tt>} list
        \item task {\tt<}filter{\tt>} start {\tt<}mods{\tt>}
        \item task {\tt<}filter{\tt>} stop {\tt<}mods{\tt>}
        \item task {\tt<}filter{\tt>} done {\tt<}mods{\tt>}
    \end{itemize} \pause

    To get an overview, take a look at the \href{http://taskwarrior.org/download/task-2.5.1.ref.pdf}{cheat sheet} (pdf, 145kB) \\
    (or come over and grab a printed copy).
\end{frame}

\begin{frame}[fragile]\frametitle{task {\tt<}filter{\tt>} command {\tt<}mods{\tt>}}
    \vfill
    \begin{itemize}
        \item Is the basic usage of all task related \textbf{write} commands.
        \item Write commands can operate on one task or a group of tasks or even on all tasks.
        \item Every command may be \textbf{abbreviated} up to the minimum that is necessary to identify a single command.
        \item Filters can be anything from nothing to simple IDs to regular expressions or Boolean constructs.
        \item Modifications can be either a change of description, a change of dates or anything else that changes a task.
        \item In our simple example we already used the write commands \textbf{add}, \textbf{done}, \textbf{start} and \textbf{stop}.
    \end{itemize}
\end{frame}

\begin{frame}[fragile]\frametitle{Scripts}
    \vfill
    \begin{lstlisting}
# Scripts shipped with Taskwarrior
ls /usr/local/share/doc/task/scripts/*\end{lstlisting}

\begin{lstlisting}
# Commandline completion tabtabtabtabtabtab ;-)
source /usr/local/share/doc/task/scripts/bash/task.sh

# Make it persistent
echo source /usr/local/share/doc/task/scripts/bash/task.sh >> .bashrc\end{lstlisting}

\begin{lstlisting}
# Syntaxhighlighting for vim
[[ -d ~/.vim ]] || mkdir ~/.vim
cp -r /usr/local/share/doc/task/scripts/vim ~/.vim\end{lstlisting}
\end{frame}

\begin{frame}[fragile]\frametitle{Most important commands}
    \vfill
    These are the most important commands, just because I use them most ;-)

    \begin{itemize}
        \item \textbf{task {\tt<}filter{\tt>} modify} \\
        The name says it, it modifies tasks according to the filter used. \pause
        \item \textbf{task {\tt<}filter{\tt>} edit} \\
        This starts your favourite editor with the tasks you want to change. \\
        (Remember the syntax highlighting for vim?) \pause
        \item \textbf{task undo} \\
        Reverts the most recent change to a task. \pause
        \item \textbf{task help} \\
        Gives an overview of implemented commands and custom reports.
    \end{itemize}
\end{frame}

\begin{frame}[fragile]\frametitle{No kidding!}
    \vfill
    \begin{itemize}
        \item \textbf{man task (taskrc, task-sync)} \\
        Show the (almighty) man-page(s). Unlike the man-pages of many other programs they are extremely helpful and full of information and examples. \\
        Try them!
    \end{itemize}
\end{frame}

%%%%%%%%%%%%%%%%%%%%%%%%%%%%%%%%%%%%%%%%%%%%%%%%%%%%%%%%%%%%%%%%%%%%%%%%
\section{Working with dates}
%%%%%%%%%%%%%%%%%%%%%%%%%%%%%%%%%%%%%%%%%%%%%%%%%%%%%%%%%%%%%%%%%%%%%%%%

\begin{frame}[fragile]\frametitle{Dateformats (incomplete) -- from 'man taskrc'}
    \vfill
    \begin{lstlisting}
m  minimal-digit month,   for example 1 or 12
d  minimal-digit day,     for example 1 or 30
y  two-digit year,        for example 09
D  two-digit day,         for example 01 or 30
M  two-digit month,       for example 01 or 12
Y  four-digit year,       for example 2009
a  short name of weekday, for example Mon or Wed
A  long name of weekday,  for example Monday or Wednesday
b  short name of month,   for example Jan or Aug
B  long name of month,    for example January or August
V  weeknumber,            for example 03 or 37
H  two-digit hour,        for example 03 or 11
N  two-digit minutes,     for example 05 or 42
S  two-digit seconds,     for example 07 or 47\end{lstlisting}
\end{frame}

\begin{frame}[fragile]\frametitle{ISO supported}
    \vfill
    \begin{alertblock}{Defined dateformats}
        The dateformat you define, will be used in \textbf{addition} to all the standard supported \href{https://en.wikipedia.org/wiki/ISO_8601}{ISO-8601} formats.
    \end{alertblock}
\end{frame}

\begin{frame}[fragile]\frametitle{Set dateformat}
    \vfill
    \begin{lstlisting}
task show dateformat

task config dateformat YMD
task config dateformat.annotation YMD
task config dateformat.report YMD

# my dateformat once was YMD-HN

task show dateformat

grep dateformat ~/.taskrc\end{lstlisting}
\end{frame}

\begin{frame}[fragile]\frametitle{Set weekstart}
    \vfill
    \begin{lstlisting}
task show weekstart

task config weekstart Monday

task show | wc -l # nearly everything can be customized
235\end{lstlisting}
\end{frame}

\begin{frame}[fragile]\frametitle{Special dates (1)}
    \vfill
    \begin{description}
    \item[Relative wording] \hfill \\
        task \ldots due:today \\
        task \ldots due:yesterday \\
        task \ldots due:tomorrow \\
    \item[Day number with ordinal] \hfill \\
        task \ldots due:23rd \\
        task \ldots due:3wks \\
        task \ldots due:1day \\
        task \ldots due:9hrs \\
    \item[At some point or later] (sets the wait date to 1/18/2038) \hfill \\
        task \ldots wait:later \\
        task \ldots wait:someday
    \end{description}
\end{frame}

\begin{frame}[fragile]\frametitle{Special dates (2)}
    \vfill
    \begin{description}
        \item[Start / end of \ldots] (remember weekstart setting) \hfill \\
            task \ldots due:sow/eow \# week \\
            task \ldots due:soww/eoww \# workweek \\
            task \ldots due:socw/eocw \# current week \\
            task \ldots due:som/eom \# month \\
            task \ldots due:soq/eoq \# quarter \\
            task \ldots due:soy/eoy \# year \\
        \item[Next occurring weekday] \hfill \\
            task \ldots due:fri
    \end{description}
\end{frame}

\begin{frame}[fragile]\frametitle{Due and wait}
    \vfill
    \begin{lstlisting}
task add due:20161231 "Celebrate Sylvester"
task add due:Sunday "Drive home"

task list

task x modify wait:20161107

task list\end{lstlisting}
\end{frame}

\begin{frame}[fragile]\frametitle{Urgency and next}
    \vfill
    Based on your tasks attributes especially -- but not limited to -- the due date, Taskwarrior calculates an urgency value which will be used by some reports to sort the tasks. \pause

    You can increase urgency by adding the \verb=+next= tag. \pause

    This is a very complex topic and goes beyond the scope of this introductory workshop.
\end{frame}

\begin{frame}[fragile]\frametitle{Recurrence}
    \vfill
    \begin{lstlisting}
task waiting
task x modify due:eom recur:monthly

task list

task recurring

# task id changed from x (task modify) to y
# try task x edit\end{lstlisting}
\end{frame}

\begin{frame}[fragile]\frametitle{Recurrence modifiers (1)}
    \vfill
    \begin{description}
        \item[hourly] \hfill \\
            Every hour.
        \item[daily, day, 1da, 2da, \ldots] \hfill \\
            Every day or a number of days.
        \item[weekdays] \hfill \\
            Mondays, Tuesdays, Wednesdays, Thursdays, Fridays and skipping weekend days.
        \item[weekly, 1wk, 2wks, \ldots] \hfill \\
            Every week or a number of weeks.
        \item[biweekly, fortnight] \hfill \\
            Every two weeks.
        \item[monthly] \hfill \\
            Every month.
    \end{description}
\end{frame}

\begin{frame}[fragile]\frametitle{Recurrence modifiers (2)}
    \vfill
    \begin{description}
        \item[quarterly, 1qtr, 2qtrs, \ldots] \hfill \\
            Every three months, a quarter, or a number of quarters.
        \item[semiannual] \hfill \\
            Every six months.
        \item[annual, yearly, 1yr, 2yrs, \ldots] \hfill \\
            Every year or a number of years.
        \item[biannual, biyearly, 2yr] \hfill \\
            Every two years.
    \end{description}
\end{frame}

\begin{frame}[fragile]\frametitle{Recurrence based on hours}
    \vfill
    \begin{alertblock}{No alarm!}
        Nothing is wrong with setting a recurrence to hours or minutes, but please keep in mind that Taskwarrior is not and never will be a calendar application or an alarm clock.

        If you want to get notified, you are on your own.
    \end{alertblock}
\end{frame}


\begin{frame}[fragile]\frametitle{Until and entry}
    \vfill
    \begin{lstlisting}
task add due:eom recur:monthly until:20161231 "Pay installment for credit"

task add "Prepare slides for workshop"
task x modify entry:yesterday

task list\end{lstlisting}
\end{frame}

\begin{frame}[fragile]\frametitle{Holiday}
    \vfill
    \begin{alertblock}{Attention!}
        Holiday has nothing in common with the German words \textit{Ferien} or \textit{Urlaub} (this would be vacation). (Public) Holiday means \textit{Feiertag}.
    \end{alertblock} \pause

    You can add holidays by either adding them via \verb=task config= on the commandline or by adding them directly to the \verb=~/.taskrc=-File or by including an external holiday definition. \pause

    On \href{http://holidata.net/}{holidata.net} you find a growing list of holiday dates, licensed CC-BY and offered by volunteers. Service was introduced by the Taskwarrior team, who is responsible for hosting and conversion to different formats.
\end{frame}

\begin{frame}[fragile]\frametitle{Add holiday / Configure calendar}
    \vfill
    \begin{lstlisting}
task config holiday.swissnationalday.name Swiss National Day
task config holiday.swissnationalday.date 20170801

# Holiday is not highlighted by default
task cale 08 2017\end{lstlisting}

    \begin{lstlisting}
task show calendar
task config calendar.holidays full

task cale 08 2017\end{lstlisting}
\end{frame}

\begin{frame}[fragile]\frametitle{Calendar with due tasks}
    \vfill
    \begin{lstlisting}
task config calendar.holidays sparse
task config calendar.details full

task cale\end{lstlisting}
\end{frame}

%%%%%%%%%%%%%%%%%%%%%%%%%%%
\section{Getting sorted}
%%%%%%%%%%%%%%%%%%%%%%%%%%%

\begin{frame}[fragile]\frametitle{Project and subproject}
    \vfill
    \begin{lstlisting}
task x modify pro:openrheinruhr
task y modify pro:openrheinruhr.workshop
task z modify pro:private

task list\end{lstlisting}
\end{frame}

\begin{frame}[fragile]\frametitle{Projects}
    \vfill
    \begin{lstlisting}
task projects

task pro:openrheinruhr ls

task x done\end{lstlisting}
\end{frame}

\begin{frame}[fragile]\frametitle{Tags}
    \vfill
    \begin{lstlisting}
task x modify +banking
task y modify +banking

task list

task x mod -banking +oberhausen

task +oberhausen list\end{lstlisting}
\end{frame}

\begin{frame}[fragile]\frametitle{Priorities}
    \vfill
    \begin{lstlisting}
task long

task x modify pri:H # can be either (H)igh, (M)edium or (L)ow

task long\end{lstlisting}
\end{frame}

\begin{frame}[fragile]\frametitle{Annotations}
    \vfill
    \begin{lstlisting}
task x annotate "Do not forget your head"

task y annotate "Use wifes account"

task list

task y denotate "Use wifes account"\end{lstlisting}
\end{frame}

%%%%%%%%%%%%%%%%%%%%%%%%%%%%%%%%%%%%%%%%%%%%%%%%%%%%%%%%%%%%%%%%%%%%%%%%
\section{Dependencies}
%%%%%%%%%%%%%%%%%%%%%%%%%%%%%%%%%%%%%%%%%%%%%%%%%%%%%%%%%%%%%%%%%%%%%%%%

\begin{frame}[fragile]\frametitle{Dependency, part 1}
    \vfill
    \begin{lstlisting}
task add "Send letter to Fritz"

task add "Write letter"

task x modify depends:y

task blocked

task unblocked\end{lstlisting}
\end{frame}

\begin{frame}[fragile]\frametitle{Dependency, part 2}
    \vfill
    \begin{lstlisting}
task x done

task list\end{lstlisting}
\end{frame}

\begin{frame}[fragile]\frametitle{Undo}
    \vfill
    \begin{lstlisting}
task undo\end{lstlisting}
\end{frame}

\begin{frame}[fragile]\frametitle{Dependency, part 3}
    \vfill
    \begin{lstlisting}
task x,y done

task blocked\end{lstlisting}
\end{frame}

%%%%%%%%%%%%%%%%%%%%%%%%%%%%%%%%%%%%%%%%%%%%%%%%%%%%%%%%%%%%%%%%%%%%%%%%
\section{Reports}
%%%%%%%%%%%%%%%%%%%%%%%%%%%%%%%%%%%%%%%%%%%%%%%%%%%%%%%%%%%%%%%%%%%%%%%%

\begin{frame}[fragile]\frametitle{Predefined reports (from task reports), part 1}
    \vfill
    These reports were already used.

    \begin{itemize}
        \item \textbf{blocked}          Lists all blocked tasks matching the specified criteria
        \item \textbf{list}             Lists all tasks matching the specified criteria
        \item \textbf{long}             Lists all task, all data, matching the specified criteria
        \item \textbf{projects}         Shows a list of all project names used, and how many tasks are in each
        \item \textbf{recurring}        Lists recurring tasks matching the specified criteria
        \item \textbf{unblocked}        Lists all unblocked tasks matching the specified criteria
        \item \textbf{waiting}          Lists all waiting tasks matching the specified criteria
    \end{itemize}
\end{frame}

\begin{frame}[fragile]\frametitle{Predefined reports (from task reports), part 2}
    \vfill
    New ones:

    \begin{itemize}
        \item \textbf{active}           Lists active tasks matching the specified criteria
        \item \textbf{all}              Lists all tasks matching the specified criteria, including parents of recurring tasks
        \item \textbf{blocking}         Blocking tasks
        \item \textbf{burndown.daily}   Shows a graphical burndown chart, by day
        \item \textbf{burndown.monthly} Shows a graphical burndown chart, by month
        \item \textbf{burndown.weekly}  Shows a graphical burndown chart, by week
        \item \textbf{completed}        Lists completed tasks matching the specified criteria
    \end{itemize}
\end{frame}

\begin{frame}[fragile]\frametitle{Predefined reports (from task reports), part 3}
    \vfill
    And more:

    \begin{itemize}
        \item \textbf{ghistory.annual}  Shows a graphical report of task history, by year
        \item \textbf{ghistory.monthly} Shows a graphical report of task history, by month
        \item \textbf{history.annual}   Shows a report of task history, by year
        \item \textbf{history.monthly}  Shows a report of task history, by month
        \item \textbf{information}      Shows all data and metadata for specified tasks
        \item \textbf{ls}               Minimal listing of all tasks matching the specified criteria
        \item \textbf{minimal}          A really minimal listing
        \item \textbf{newest}           Shows the newest tasks
        \item \textbf{next}             Lists the most urgent tasks
    \end{itemize}
\end{frame}

\begin{frame}[fragile]\frametitle{Predefined reports (from task reports), part 4}
    \vfill
    The leftovers:

    \begin{itemize}
        \item \textbf{oldest}           Shows the oldest tasks
        \item \textbf{overdue}          Lists overdue tasks matching the specified criteria
        \item \textbf{ready}            Most urgent actionable tasks
        \item \textbf{summary}          Shows a report of task status by burndown-dailyoject
        \item \textbf{tags}             Shows a list of all tags used
    \end{itemize}

    26 reports in total (as told by \verb=task reports=)
\end{frame}

\begin{frame}[fragile]\frametitle{Test the reports}
    \vfill
    \begin{lstlisting}
task burndown.daily

task ghistory.annual
task ghistory.monthly

task history.monthly

task ls

task minimal

task summary\end{lstlisting}
\end{frame}

\begin{frame}[fragile]\frametitle{Report definitions}
    \vfill
    \begin{lstlisting}
task show report.minimal
task show report.list

task show report # to see all definitions\end{lstlisting}
\end{frame}

\begin{frame}[fragile]\frametitle{Dirks former task list}
    \vfill
    \begin{lstlisting}
echo "
report.ll.description=Dirks task list
report.ll.columns=id,project,priority,due,due.countdown,tags,description
report.ll.labels=ID,Project,Pri,Due,Countdown,Tags,Description
report.ll.sort=due+,priority-,project+,description+
report.ll.filter=status:pending
" >> ~/.taskrc

task ll\end{lstlisting}
\end{frame}

\begin{frame}[fragile]\frametitle{Set default command}
    \vfill
    \begin{lstlisting}
task show default

task config default.command ll
task\end{lstlisting}
\end{frame}

%%%%%%%%%%%%%%%%%%%%%%%%%%%%%%%%%%%%%%%%%%%%%%%%%%%%%%%%%%%%%%%%%%%%%%%%
\section{Filtering}
%%%%%%%%%%%%%%%%%%%%%%%%%%%%%%%%%%%%%%%%%%%%%%%%%%%%%%%%%%%%%%%%%%%%%%%%

\begin{frame}[fragile]\frametitle{Filtering in general}
    \vfill
    You can filter for any modifier. If you don't use a modifier description is searched for the term, which may be a regular expression, on the command line. Filters may be combined.

    The following attribute modifiers maybe applied as well. Names in brackets can be used alternatively.

    So a filter can look like \verb=attribute.modifier:value=.

    \begin{itemize}
        \item before, after
        \item none, any
        \item is (equals), isnt (not)
        \item has (contains), hasnt
        \item startswith (left), endswith (right)
        \item word, noword
    \end{itemize}
\end{frame}

\begin{frame}[fragile]\frametitle{Searches}
    \vfill
    \begin{lstlisting}
task

task pay

task /[Pp]ay/\end{lstlisting}
\end{frame}

\begin{frame}[fragile]\frametitle{Attribute modifiers}
    \vfill
    \begin{lstlisting}
task due.before:20161130

task project.not:taskwarrior\end{lstlisting} \pause

    \begin{lstlisting}
task project:openrheinruhr +banking
task status:completed project:openrheinruhr
task status:completed project:openrheinruhr completed

task show report.ll.filter\end{lstlisting}
\end{frame}

\begin{frame}[fragile]\frametitle{Or \ldots}
    \vfill
    \begin{lstlisting}
task list

task \( pro:taskwarrior or pro:private \) list
# Brackets must be escaped for the shell\end{lstlisting}
\end{frame}

\begin{frame}[fragile]\frametitle{Search configuration}
    \vfill
    \begin{lstlisting}
task show search

task show regex\end{lstlisting}
\end{frame}

\begin{frame}[fragile]\frametitle{Filter in reports}
    \vfill
    \begin{lstlisting}
task show filter\end{lstlisting}
\end{frame}

\begin{frame}[fragile]\frametitle{Contexts}
    \vfill
       Context is a user-defined filter, which is automatically applied to all commands that filter the task list. In particular, any report command will have its result affected by the current active context.

    \begin{itemize}
        \item \verb+task context define <name> <filter>+
        \item \verb+task context delete <name>+
        \item \verb+task context <name>+ -- sets active context
        \item \verb+task context show+ -- shows active context
        \item \verb+task context list+ -- lists available contexts
        \item \verb+task context none+ -- clears active context
    \end{itemize}
\end{frame}

%%%%%%%%%%%%%%%%%%%%%%%%%%%%%%%%%%%%%%%%%%%%%%%%%%%%%%%%%%%%%%%%%%%%%%%%
\section{Miscellanous}
%%%%%%%%%%%%%%%%%%%%%%%%%%%%%%%%%%%%%%%%%%%%%%%%%%%%%%%%%%%%%%%%%%%%%%%%

\begin{frame}[fragile]\frametitle{Virtual Tags (1)}
    \vfill
    \begin{itemize}
        \item \textbf{ACTIVE}       -- Task is started
        \item \textbf{ANNOTATED}    -- Task has annotations
        \item \textbf{BLOCKED}      -- Task is blocked
        \item \textbf{BLOCKING}     -- Task is blocking
        \item \textbf{CHILD}        -- Task has a parent
        \item \textbf{COMPLETED}    -- Task has completed status
        \item \textbf{DELETED}      -- Task has deleted status
        \item \textbf{DUE}          -- Task is due
        \item \textbf{LATEST}       -- Task is the newest added task
        \item \textbf{MONTH}        -- Task is due this month
        \item \textbf{ORPHAN}       -- Task has any orphaned UDA values
        \item \textbf{OVERDUE}      -- Task is overdue
        \item \textbf{PARENT}       -- Task is a parent
        \item \textbf{PENDING}      -- Task has pending status
    \end{itemize}
\end{frame}

\begin{frame}[fragile]\frametitle{Virtual Tags (2)}
    \vfill
    \begin{itemize}
        \item \textbf{PRIORITY}     -- Task has a priority
        \item \textbf{PROJECT}      -- Task has a project
        \item \textbf{READY}        -- Task is actionable
        \item \textbf{SCHEDULED}    -- Task is scheduled
        \item \textbf{TAGGED}       -- Task has tags
        \item \textbf{TODAY}        -- Task is due today
        \item \textbf{TOMORROW}     -- Task is due sometime tomorrow
        \item \textbf{UDA}          -- Task has any UDA values
        \item \textbf{UNBLOCKED}    -- Task is not blocked
        \item \textbf{UNTIL}        -- Task expires
        \item \textbf{WAITING}      -- Task is waiting
        \item \textbf{WEEK}         -- Task is due this week
        \item \textbf{YEAR}         -- Task is due this year
        \item \textbf{YESTERDAY}    -- Task was due sometime yesterday
    \end{itemize}
\end{frame}

\begin{frame}[fragile]\frametitle{This is by far not all}
    \vfill
    \begin{description}
        \item[task log] \hfill \\
            for logging a task after it is already done.
        \item[task diag] \hfill \\
            to help support for diagnostic purpose.
        \item[UDA] \hfill \\
            User defined attributes.
        \item[\ldots] \hfill \\
            and many more!
    \end{description}
\end{frame}

\begin{frame}[fragile]\frametitle{Questions?}
    \vfill
    \begin{center}
        \includegraphics[width=6.4cm,height=7.5cm]{task_logo.png}
    \end{center}
\end{frame}

%%%%%%%%%%%%%%%%%%%%%%%%%%%%%%%%%%%%%%%%%%%%%%%%%%%%%%%%%%%%%%%%%%%%%%%%
\section{Epilog}
%%%%%%%%%%%%%%%%%%%%%%%%%%%%%%%%%%%%%%%%%%%%%%%%%%%%%%%%%%%%%%%%%%%%%%%%

\begin{frame}[fragile]\frametitle{Getting Help}
    \vfill
    There are several ways of getting help:

    \begin{itemize}
        \item Submit your details to our \href{https://answers.tasktools.org}{Q \& A site}, then wait patiently for the community to respond.
        \item Email us at \href{mailto:support@taskwarrior.org}{support@taskwarrior.org}, then wait patiently for a volunteer to respond.
        \item Join us IRC in the \#taskwarrior channel on Freenode.net, and get a quick response from the community, where, as you have anticipated, we will walk you through the checklist above.
        \item Even though Twitter is no means of support, you can get in touch with \href{https://twitter.com/taskwarrior}{@taskwarrior}.
        \item We have a \href{https://groups.google.com/forum/\#!forum/taskwarrior-user}{User Mailinglist} which you can join anytime to discuss about Taskwarrior and techniques.
        \item The \href{https://groups.google.com/forum/\#!forum/taskwarrior-dev}{Developer Mailinglist} is focussing on a more technical oriented audience.
    \end{itemize}
\end{frame}

\begin{frame}[fragile]\frametitle{Thanks for your patience!}
    \vfill
    \begin{center}
        Dirk Deimeke, Taskwarrior-Team, 2016, \href{https://creativecommons.org/licenses/by/4.0/}{CC-BY}

        \href{mailto:dirk@deimeke.net}{dirk@deimeke.net} -- \href{https://d5e.org/}{d5e.org}
    \end{center}
\end{frame}

\end{document}
